\section{Ejercicio 1: Control de bombas de agua}
Se pretende implementar un sistema controlador de dos bombas de agua cuya función es mantener el nivel de agua de un tanque entre un rango marcado por dos 
sensores de nivel.
Ambas bombas estarán encendidas cuando el nivel de agua esté por debajo del mínimo (sensor I=0), y apagadas cuando este supere el máximo (sensor S=1).
En un nivel intermedio (I=1 y S=0), solo una de las bombas se encontrará trabajando, y aquella en hacerlo será la última en haber permanecido apagada mientras la otra 
bomba estaba en funcionamiento; es decir, si en determinado momento está trabajando solo la bomba 1 (B1), la próxima vez que se dé la condición I=1 y S=0, se pondrá en 
funcionamiento la bomba 2 (B2), y vice versa.
Es en esta característica de \"memoria\" en donde el planteo de una solución a la problemática mediante una máquina de estados, se vuelve considerable.



\subsection{Diseño de Máquina de Estados}
Se propone como diseño para la máquina de estados aquella que sigue el diagrama de la Figura \ref{fig:fsm_state_chart_ex1}.
Cabe destacar que para la combinación de entradas I=0 y S=1, que resulta imposible en la aplicación del circuito a la realidad, dado que el sensor I se encuentra por 
debajo del sensor S, se decidió apagar ambas bombas por seguridad.
\begin{figure}[H]
    \centering
    \includegraphics[width=0.7\textwidth]{../EJ1/Recursos/fsm_state_chart.jpg}
    \caption{Diagrama de estados para sistema de control de bombas.}
    \label{fig:fsm_state_chart_ex1}
\end{figure}

Como se puede observar, la máquina de estados planteada corresponde a una implementación de Mealy.
La decisión de utilizar esta implementación por sobre una de Moore, es que una aplicación estricta de la segunda, donde la salida dependiera únicamente del estado actual,
hubiera supuesto el uso de 6 estados, resultando altamente ineficiente en comparación a la solución elegida.\\
\\
La Tabla \ref{tab:truth_table_ex1} representa la tabla de verdad correspondiente a la máquina de estados en cuestión, donde \"y\" es el estado actual, \"I\" y \"S\" las entradas,
\"Y\" el siguiente estado a partir del próximo clock, y \"B1\" y \"B2\" las salidas asincrónicas. 
\begin{table}[H]
    \centering
    \begin{tabular}{ccc|ccc}
    \textbf{I} & \textbf{S} & \textbf{y} & Y & B1 & B2 \\ \hline
    0          & 0          & 0          & 1 & 1  & 1  \\
    0          & 0          & 1          & 0 & 1  & 1  \\
    0          & 1          & 0          & 0 & 0  & 0  \\
    0          & 1          & 1          & 1 & 0  & 0  \\
    1          & 0          & 0          & 0 & 1  & 0  \\
    1          & 0          & 1          & 1 & 0  & 1  \\
    1          & 1          & 0          & 1 & 0  & 0  \\
    1          & 1          & 1          & 0 & 0  & 0 
    \end{tabular}
    \caption{Tabla de verdad para máquina de estados de control de bombas.}
    \label{tab:truth_table_ex1}
    \end{table}

De la observación de las salidas, se infiere que los circuitos lógicos para implementarlas son los de las Figuras \ref{fig:Y_logic_circuit_ex5}, 
\ref{fig:B1_logic_circuit_ex5} y \ref{fig:B2_logic_circuit_ex5}.
Luego, la máquina de estados estará conformada por el circuito de la Figura \ref{fig:fsm_circtuit_ex5}.
\begin{figure}[H]
    \centering
    \includegraphics[width=0.4\textwidth]{../EJ1/Recursos/Y_logic_circuit.jpg}
    \caption{Circuito lógico para la entrada del Flip-Flop D.}
    \label{fig:Y_logic_circuit_ex5}
\end{figure}
\begin{figure}[H]
    \centering
    \includegraphics[width=0.4\textwidth]{../EJ1/Recursos/B1_logic_circuit.jpg}
    \caption{Circuito lógico para la salida de la bomba 1.}
    \label{fig:B1_logic_circuit_ex5}
\end{figure}
\begin{figure}[H]
    \centering
    \includegraphics[width=0.5\textwidth]{../EJ1/Recursos/B2_logic_circuit.jpg}
    \caption{Circuito lógico para la salida de la bomba 2.}
    \label{fig:B2_logic_circuit_ex5}
\end{figure}
\begin{figure}[H]
    \centering
    \includegraphics[width=0.7\textwidth]{../EJ1/Recursos/fsm_circuit.jpg}
    \caption{Circuito de la máquina de estados.}
    \label{fig:fsm_circtuit_ex5}
\end{figure}



\subsection{Simulaciones en Verilog}
Se desea comprobar el funcionamiento de la máquina de estados a nivel l\'ogico mediante una simulaci\'on en Verilog, para lo cual se emplean
dos metodolog\'ias de dise\~no.
Por un lado, puede utilizarse un dise\~no en Verilog que determine si la m\'aquina est\'a bien diagramada, utilizando para ello el bloque procedural case.
Por otro lado, para determinar si la implementaci\'on l\'ogica de la m\'aquina puede funcionar, debe emplearse un dise\~no a nivel compuertas de los m\'odulos en Verilog,
para esto \'ultimo se divide el problema inicialmente en tres bloques, los flip flops, la l\'ogica combinacional que produce el pr\'oximo estado y la l\'ogica de salida.
Finalmente, un cuarto bloque o m\'odulo describe la m\'aquina interconectando los m\'odulos mencionados para producir el comportamiento esperado.

En la Figura \ref{fig:gtkwave_sim_ex5} se observa el comportamiento de la máquina de estados ante los diferentes escenarios de prueba.
Se puede apreciar como el comportamiento de la cantidad de bombas que deben encenderse es el correcto, es decir, cuando los dos sensores mandan señal \textit{low}, ambas 
bombas se encuentran encendidas, cuando, en cambio, lo que ingresa son dos señales \textit{high}, ambas bombas se apagan.
Cuando el agua se encuentra entre los dos sensores, la bomba que se enciende se va alternando, y, finalmente, ante la situación no esperada, se apagan las bombas por seguridad.
\begin{figure}[H]
    \centering
    \includegraphics[width=\textwidth]{../EJ1/Recursos/gtkwave_sim.png}
    \caption{Salida de la simulación de la máquina de estados en GTKWave.}
    \label{fig:gtkwave_sim_ex5}
\end{figure}

La misma simulación puede realizarse observando los resultados desde linea de comandos, y la salida es la siguiente:
\begin{lstlisting}
    Running simulation!
    VCD info: dumpfile bin/output.vcd opened for output.
    Testing I=0 S=0. Output: 11. Next state is 1. Current state is 0
    Testing I=1 S=0. Output: 10. Next state is 1. Current state is 1
    Testing I=1 S=1. Output: 00. Next state is 0. Current state is 1
    Testing I=1 S=0. Output: 01. Next state is 0. Current state is 0
    Testing I=0 S=0. Output: 11. Next state is 1. Current state is 0
    Testing I=1 S=0. Output: 10. Next state is 1. Current state is 1
    Testing I=0 S=0. Output: 11. Next state is 0. Current state is 1
    Testing I=1 S=1. Output: 00. Next state is 1. Current state is 0
    Testing I=0 S=0. Output: 11. Next state is 0. Current state is 1
    Testing I=1 S=0. Output: 01. Next state is 0. Current state is 0
    Testing impossible situation. Output: 00. Next state is 0. Current state is 0
    Testing I=1 S=0. Output: 01. Next state is 0. Current state is 0
\end{lstlisting}



\subsection{Diseño en PCB}



\subsection{Resultados}



\subsection{Conclusiones}